\documentclass[a4paper,11pt]{article}
\usepackage[frenchb]{babel}
\usepackage[T1]{fontenc}           % for hyphenations to work
\usepackage{amsmath,amssymb,amsfonts,textcomp}
\usepackage{color}
\usepackage{graphicx}
\usepackage{tabularx}
\usepackage[frenchb]{minitoc}
\usepackage{multirow}
\usepackage[round,authoryear]{natbib}
\usepackage{fancyhdr}
\usepackage{afterpage}
\usepackage{longtable}
\usepackage{soul}
\usepackage[latin1]{inputenc}
\usepackage{array}
\usepackage{subfigure}
\usepackage{hhline}
\usepackage{url}
\usepackage[table]{xcolor}






\usepackage[colorlinks=true, urlcolor=darkblue,pagebackref]{hyperref}

%Ajouts Anthony
\usepackage{version}
\DeclareMathAlphabet{\mathscr}{OT1}{pzc}%
                                 {m}{it}
\usepackage[mathcal]{eucal}
%\DeclareSymbolFont{calletters}{OMS}{cmsy}{m}{n}
%\DeclareSymbolFontAlphabet{\mathcal}{calletters}

 \usepackage[acronym,footnote]{glossaries} % make a separate list of acronyms




%\renewcommand{\chaptermark}[1]{\markboth{\chaptername\ \thechapter. #1}{}}
%\renewcommand{\sectionmark}[1]{\markright{\thesection. #1}}
%% Setting up pagestyles for ``fancy''
\chead{}
\rhead[]{\nouppercase{\rightmark}}
\lhead[\nouppercase{\leftmark}]{}
\renewcommand{\headrulewidth}{0.5pt}

\newenvironment{palat}
  {\fontfamily{lasy}\fontshape{sl}\selectfont}{}


%\footrulewidth 0.5pt
\lfoot{}
\cfoot{\thepage}
\rfoot{}

% watch a movie with one frame on each page
%\rfoot{\setlength{\unitlength}{1mm} 
%\begin{picture}(0,0) 
%\put(5,0){\includegraphics{pic\thepage.ps}} 
%\end{picture}} 

% �quivalent ? \usepackage{palatino}
\usepackage{mathpazo} 
\usepackage[scaled=.95]{helvet} 
\usepackage{courier} 

\usepackage[twoside]{geometry}
\setlength{\oddsidemargin}{1cm}
\setlength{\evensidemargin}{0cm}
\addtolength{\textheight}{1cm}
\addtolength{\headsep}{0.3cm}
\addtolength{\footskip}{0.3cm}
\usepackage{layout}
%\newlength{\oldtextwidth}
%\setlength{\oldtextwidth}{\textwidth}
%\usepackage[margin=1cm]{geometry}
%\usepackage[margin=2.5cm]{geometry}
\usepackage{setspace}

\usepackage{boxedminipage}
\usepackage{rotating}
\newcommand\vertical[1]{\begin{sideways}#1\end{sideways}}


\newcommand{\mychaptermark}[1]{\markboth{ Chapter \thechapter.\  #1}{}}


\usepackage[width=.95\textwidth,font={small,it},labelfont=bf]{caption}
%\renewcommand{\captionfont}{\small\itshape}
\setlength{\fboxsep}{5pt}

\usepackage{parskip}
\setlength{\parindent}{1.5em}
% re-d�finition des marges
%\textwidth = 15 cm
%\textheight = 22 cm
%\oddsidemargin = 1.0 cm
%\evensidemargin = 0.0 cm
%\topmargin = 0.0 cm
%\headheight = 0.0 cm
%\headsep = 1.0 cm
%\setlength{\parskip}{0.5em}
%\parindent = 0.6 cm
%\abovecaptionskip = 0 cm

% line badness cooking
%\tolerance 1414 
%\hbadness 1414 
%\emergencystretch 1.5em 
%\hfuzz 0.3pt 
%\widowpenalty=10000 
%\vfuzz \hfuzz 
%\raggedbottom 

%\prefacename Preface Pr�face 
%\refname a References R�f�rences 
%\abstractname Abstract R�sum� 
%\bibname b Bibliography Bibliographie 
%\chaptername Chapter Chapitre 
%\appendixname Appendix Annexe 
%\contentsname Contents Table des mati�res 
%\listfigurename List of Figures Table des figures 
%\listtablename List of Tables Liste des tableaux 
%\indexname Index Index 
%\figurename Figure F I G. 
%\tablename Table TAB . 
%\partname Part partie 
%\enclname encl P. J. 
%\ccname cc Copie ? 
%\headtoname To 
%\pagename Page page 
%\seename see voir 
%\alsoname see also voir aussi 
\addto{\captionsfrench}{\renewcommand*{\listfigurename}{Liste des illustrations}} 

%\renewcommand\floatpagefraction{.9}
%\renewcommand\topfraction{.9}
%\renewcommand\bottomfraction{.9}
%\renewcommand\textfraction{.1}   
%\setcounter{totalnumber}{50}
%\setcounter{topnumber}{50}
%\setcounter{bottomnumber}{50}

%\clubpenalty=9999
%\widowpenalty=9999

%\usepackage[plainpages=false,pdfpagelabels,hypertexnames=false]{hyperref}
\usepackage{hyperref}
%\hypersetup{plainpages=false,pdfpagelabels,hypertexnames=false,colorlinks=true, linkcolor=black, filecolor=black, pagecolor=black, urlcolor=black, citecolor=black}
\hypersetup{	pagebackref=true,	
			backref=true,
			plainpages=false,
			pdfpagelabels,
			hypertexnames=false,
			colorlinks=true, 
			linkcolor=blue, 
			filecolor=blue, 
			pagecolor=blue, 
			urlcolor=blue, 
			citecolor=blue,
			pdftitle={Th\`{e}se},pdfauthor={Nom de l'auteur},pdfsubject={LaTex}}

\newcommand\textsubscript[1]{\ensuremath{_{\text{#1}}}}
\everymath{\displaystyle}

%\newcommand\todo[1]{[\textcolor[rgb]{0,0,0}{\textbf{? faire:} \textit{#1}}]}
\newcommand\todo[1]{ [\textcolor[rgb]{0.5,0,0}{\textbf{? faire:} #1}] }
%\newcommand\todo[1]{}
%\newcommand\tocite[1]{[\textcolor[rgb]{0,0,0}{\textbf{? citer:} \textit{#1}}]}
\newcommand\tocite[1]{[\textcolor[rgb]{0,0.5,0}{\textbf{? citer:} #1}]}
%\newcommand\tocite[1]{}

%\renewcommand\label[1]{[\textcolor[rgb]{0,0,0.5}{\textbf{label:} #1\label{#1}}]}
\newcommand\quotes[1]{\guillemotleft\nobreak\hspace{1.5pt}#1\nobreak\hspace{1.5pt}\guillemotright}
\newcommand\cfigurex[4]
{
	\begin{figure}[hbt!]
	\centering
	\includegraphics[width=#4\textwidth]{#1}
	\caption[#2]{#3}
	\end{figure}
}

\newcommand\cfigure[3]
{
	\begin{figure}[hbt!]
	\centering
	\includegraphics[width=#3\textwidth]{#1}
	\caption{#2}
	\end{figure}
}

\newcommand\cfigurebox[3]
{
	\begin{figure}[hbt!]
	\centering
	\begin{boxedminipage}{#3\textwidth}
	\includegraphics[width=\textwidth]{#1}
	\end{boxedminipage}
	\caption{#2}
	\end{figure}
}

\newcommand\ctablex[4]%
{%
	\begin{table}[hbt]
	\centering
	\begin{tabular}{ #1}
	\hline
	#4
	\hline
	\end{tabular}
	\caption{\label{#3} #2}
	\end{table}
}

\newcommand\ctablexx[5]%
{%
	\begin{table}[hbt]
	\centering
	\begin{tabular}{ #1}
	\hline
	#5
	\hline
	\end{tabular}
	\caption[#2]{\label{#4} #3}
	\end{table}
}

\def\max{\mathop{\hbox{max}}}
\def\argmax{\mathop{\hbox{argmax}}}
\def\min{\mathop{\hbox{min}}}
\def\argmin{\mathop{\hbox{argmin}}}
\def\nb{\mathop{\hbox{nb}}}

%\newcommand{\citep}[1]{\cite{#1}}
%\newcommand{\citet}[1]{\cite{#1}}
\renewcommand{\cite}[1]{\citep{#1}}
%\newcommand{\refp}[1]{\ref{#1}, page \pageref{#1}}
%\newcommand{\refp}[1]{\ref{#1}}

%\newcommand\chapterx[1]{\chapter*{#1}\addcontentsline{toc}{chapter}{#1}}
%\newcommand\sectionx[1]{\section*{#1}\addcontentsline{toc}{section}{#1}}
%\newcommand\chapterx[1]{\chapter{#1}}
%\newcommand\sectionx[1]{\section{#1}}
%\newcommand\subsectionx[1]{\section{#1}}

%\newcommand{\etc}{, etc.}
\newcommand{\etc}{...}
%\setcounter{tocdepth}{5}
%\hyphenation{sur-ap-pren-tis-sage}
\pagestyle{plain}
\newcommand{\postheader}{
%\renewcommand{\minitoc}{}
%\doublespacing
%\addcontentsline{toc}{chapter}{Liste des tableaux}
\pagestyle{fancy}
%\bibliographyunit[\chapter] 
%\defaultbibliography{biblio-exact} 
%\defaultbibliographystyle{lia-these-fr} 
%\begin{bibunit}
}

%\usepackage{bibunits}
%\newcommand\chapterbib{}
%\usepackage[sectionbib]{bibunits}
%\newcommand\chapterbib{\putbib}

\newcommand\postdocument{
\cleardoublepage
\addstarredchapter{Liste des illustrations}
\listoffigures
\cleardoublepage
\addstarredchapter{Liste des tableaux}
\listoftables 
%\bibliographystyle{apalike-fr}
%\bibliographystyle{plain}
\cleardoublepage
\addstarredchapter{Bibliographie}

%\putbib[biblio-exact]
\bibliographystyle{lia-these-fr}
\bibliography{../../biblio}

%\end{bibunit}
%\begin{bibunit}
%\renewcommand*{\bibname}{Publications Personnelles}
%\addstarredchapter{Publications Personnelles}
%\nocite{*}
%\putbib[biblio-perso]
%\end{bibunit}


}


\title{How to generate I-Vectors with ALIZE}
\author{\small \textit{anthony.larcher@i2r.a-star.edu.sg}}

\begin{document}

\maketitle

\tableofcontents

\postheader

\newpage
\section{Total Variability paradigm}
% theorie et references
Initially introduced for speaker recognition, i-vectors \cite{Dehak11_a} have become very popular in the field of speech processing and recent publications show that they are also reliable for text-dependent speaker verification \cite{Larcher12_a} language recognition \cite{Martinez11} and speaker diarization \cite{Franco10}.
%
I-vectors convey the speaker characteristic among other information such as transmission channel, acoustic environment or phonetic content of the speech segment.

Detailed descriptions of the Total Variability paradigm could be found in \cite{Dehak11_a,Martinez11,Kanagasundaram11}. The i-vector extraction could be seen as a probabilistic compression process that reduces the dimensionality of speech-session super-vectors according to a linear-Gaussian model. The speaker- and channel-dependent super-vector $\mathcal{M}_{(s,h)}$ of concatenated Gaussian Mixture Model (GMM) means is projected in a low dimensionality space, named Total Variability space, as follows
%
\begin{equation}
\mathcal{M}_{(s,h)}=\textbf{m}+\textbf{T}\textbf{w}_{(s,h)}
\label{I_Vector_modelling}
\end{equation}
%
where $\textbf{m}$ is the mean super-vector of a gender-dependent Universal Background Model (UBM), $\textbf{T}$ is called Total Variability matrix and $\textbf{w}_{(s,h)}$ is the resulting i-vector.

\newpage
\section{ComputeJFAStat}
%
ComputeJFAStat computes Baum-Welch statistics for a list of files given a Gaussian Mixture Model.

\subsection{Inputs-Outputs}
%
\begin{description}
\item[Universal Background Model] Baum-Welch statistics are computed over this Universal background Model.
%
\item[Index file] the list of files to process is given in an index file (NDX). NDX file should be ascii file with one file per line.
%
\item[Null Order Statistics per Session] will not be used for TotalVariability estimation.
%
\item[Null Order Statistics per Speaker] this matrix contains zero order Baum-Welch statistics for each file. Dimensions of this matrix are number of files $\times$ number of distributions of the UBM.
%
\item[First Order Statistics per Session] will not be used for TotalVariability estimation.
%
\item[First Order Statistics per Speaker] this matrix contains zero order Baum-Welch statistics for each file. Dimensions of this matrix are number of files $\times$ dimension of super vector (which is equal to the number of distributions of the UBM multiplied by the dimension of acoustic features).
\end{description}

\newpage
\subsection{ComputeJFAStat options}

\begin{table}[!h]\footnotesize
\begin{center}
\begin{tabular}{m{4cm}m{4cm}m{7cm}}
Option & Value & function \tabularnewline
\hline
\hline
featureServerMemAlloc & integer & Specify the memory to allocate for the ALIZE buffer \tabularnewline
featureServerMode & FEATURE$\_$WRITABLE & Specify the read/write mode of the featureServer (read-only by default) \tabularnewline
featureServerBufferSize & ALL$\_$FEATURES & Algorithm Buffer \tabularnewline
minLLK & integer & Specify minimum likelihood values \tabularnewline
maxLLK & integer & Specify maximum likelihood values \tabularnewline
\hline
labelFilesPath & path & directory to load label files, could be relative or absolute\tabularnewline
featureFilesPath & path & directory to load feature files, could be relative or absolute\tabularnewline
matrixFilesPath & path & directory to load and save matrices model, could be relative or absolute \tabularnewline
mixtureFilesPath & path & directory to load GMM model, could be relative or absolute \tabularnewline
\hline
loadMixtureFileFormat & RAW | XML & format of the GMM to load\tabularnewline
loadMatrixFormat & DB | DT & format of the matrix to load, could be binary (DB) or text (DT)\tabularnewline
saveMatrixFormat & DB | DT & format of the matrix to save, could be binary (DB) or text (DT)\tabularnewline
loadFeatureFileFormat & SPRO4 | HTK | RAW & ALIZE can read different format of features including HTK, SPRO and RAW binary files\tabularnewline
\hline
loadMixtureFileExtension & string & extension of GMM files to load\tabularnewline
loadMatrixFilesExtension & string & extension of Matrices files to load \tabularnewline
saveMatrixFilesExtension & string & extension of Matrices files to save\tabularnewline
loadFeatureFileExtension & string & extension of feature files to load\tabularnewline
\hline
ndxFilename & string & name of the index file (NDX) containing the list of files to process\tabularnewline
inputWorldFilename & string & name of the Universal Background Model to be used for statistics computation.\tabularnewline
nullOrderStatSpeaker & string & name of the file containing null order Baum-Welch statistics per speaker (sum of all sessions for one speaker) \tabularnewline
nullOrderStatSession & string & name of the file containing null order Baum-Welch statistics per session\tabularnewline
firstOrderStatSpeaker & string & name of the file containing first order Baum-Welch statistics per speaker (sum of all sessions for one speaker)\tabularnewline
firstOrderStatSession & string & name of the file containing first order Baum-Welch statistics per session \tabularnewline
\hline
addDefaultLabel & boolean & add a default label in case no label is found for a feature file\tabularnewline
defaultLabel & string & if no label file is found, all features from this file will be given the defaultlabel\tabularnewline
labelSelectedFrames & string & feature label to use for statistics computation\tabularnewline
frameLength & float & period of the frame extraction given in seconds\tabularnewline
\hline
\end{tabular}
\end{center}
\end{table}

\section{Total Variability}
%
Total Variaiblity matrix can be computed by using EigenVoice program. However, some of the local variables used for JFA estimation are not used for Total Variability estimation and strongly increase the memory requirement. Total Variability is then another version of the EigenVoice program optimized to process a large number of sessions.
However, optimization is still balancing between computational time and memory usage and TotalVariability may still require a large amount of memory.

\subsection{Mandatory inputs/outputs}
%
\begin{description}
\item[Universal Background Model] the UBM on which statistics have been computed. The super-vector of the co-variance coefficients from the UBM will be used by TotalVariability.
\item[TotalVariability matrix] output of this executable, TotalVariability matrix contains the main eigenvectors of the Total Variability  space.
\end{description}

\subsection{Optional inputs/outputs}
%
Zero and first order Baum-Welch statistic matrices could be computed directly by the TotalVariability program. Those inputs are then not mandatory. However, considering that the quantity of files to process is important, we recommend to split the list of files to process and to parallelize the computation by using scripting and to concatenate eventually the matrices by using tools provided in ALIZE/LIA\_RAL.

\begin{description}
%
\item[Null Order Statistics] this matrix contains zero order Baum-Welch statistics for each file. Dimensions of this matrix are number of files $\times$ number of distributions of the UBM.
%
\item[First Order Statistics] this matrix contains zero order Baum-Welch statistics for each file. Dimensions of this matrix are number of files $\times$ dimension of super vector (which is equal to the number of distributions of the UBM multiplied by the dimension of acoustic features).
\end{description}

\newpage
\subsection{TotalVariability options}
%
\begin{table}[!htbf]\scriptsize
\begin{center}
\begin{tabular}{m{4cm}m{4cm}m{7cm}}
Option & Value & function \tabularnewline
\hline
\hline
numThread & integer & number of thread to run in parallel. Only available if LIA\_RAL is compiled with pThread library (see ALIZE FAQ for more information) \tabularnewline
minLLK & integer & Specify minimum likelihood values \tabularnewline
maxLLK & integer & Specify maximum likelihood values \tabularnewline
\hline
labelFilesPath & path & directory to load label files, could be relative or absolute\tabularnewline
featureFilesPath & path & directory to load feature files, could be relative or absolute\tabularnewline
matrixFilesPath & path & directory to load and save matrices model, could be relative or absolute \tabularnewline
mixtureFilesPath & path & directory to load GMM model, could be relative or absolute \tabularnewline
\hline
loadMixtureFileFormat & RAW | XML & format of the GMM to load\tabularnewline
loadMatrixFormat & DB | DT & format of the matrix to load, could be binary (DB) or text (DT)\tabularnewline
saveMatrixFormat & DB | DT & format of the matrix to save, could be binary (DB) or text (DT)\tabularnewline
loadFeatureFileFormat & SPRO4 | HTK | RAW & ALIZE can read different format of features including HTK, SPRO and raw binary files\tabularnewline
\hline
loadMixtureFileExtension & string & extension of GMM files to load\tabularnewline
saveMatrixFilesExtension & string & extension of GMM files to save\tabularnewline
loadFeatureFileExtension & string & extension of feature files to load\tabularnewline
\hline
addDefaultLabel & boolean & add a default label in case no label is found for a feature file\tabularnewline
defaultLabel & string & if no label file is found, all features from this file will be given the defaultlabel\tabularnewline
labelSelectedFrames & string & feature label to use for statistics computation\tabularnewline
frameLength & float & period of the frame extraction given in seconds\tabularnewline
\hline
ndxFilename & string & name of the index file (NDX) containing the list of files to process\tabularnewline
inputWorldFilename & string & name of the Universal Background Model to be used for statistics computation.\tabularnewline
loadAccs & boolean & if true, then load existing statistic matrices, if false the statistics are first computed\tabularnewline
nullOrderStatSpeaker & string & name of the file containing null order Baum-Welch statistics per file (in TotalVariability framework, speaker refer to one only file) \tabularnewline
firstOrderStatSpeaker & string & name of the file containing first order Baum-Welch statistics per file (in TotalVariability framework, speaker refer to one only file)\tabularnewline
\hline
nbIt & integer & number of EM iterations to perform \tabularnewline
totalVariabilityNumber & integer & rank of the TotalVariability matrix \tabularnewline
loadInitTotalVariabilityMatrix & boolean & if true, load an existing matrix for initialization\tabularnewline
randomInitLaw & normal | uniform &  distribution used to randomly initialize the matrix\tabularnewline
initTotalVariabilityMatrix & string & name of the initialization matrix to load\tabularnewline
saveInitT & boolean & save the initialization matrix \tabularnewline
totalVariabilityMatrix & string & name of the file to store the final TotalVariability matrix \tabularnewline
saveAllTVMatrices & boolean & if true, save the TotalVariability matrix after each iteration \tabularnewline
minDivergence & boolean & use minimum divergence criteria to update T matrix\tabularnewline
meanEstimate & string & name of the new mean estimate after minimum divergence \tabularnewline
checkLLK & boolean &  if true compute the likelihood after each iteration (Likelihood is computed by creating a GMM for each session and has not been optimized yet)\tabularnewline
computeLLK & integer & number of sessions to use to compute the likelihood after each iteration \tabularnewline
\hline
\end{tabular}
\end{center}
\end{table}

\newpage
\section{TrainTarget}

Extract i-Vectors for a list of files.

\subsection{Input-Output}

Baum-Welch statistics have to be computed first by using ComputeJFAStats.
I-Vectors can be extracted by using the same configuration used for JFA TrainTarget.
However, TrainTarget run in mode "\textit{iVector}" is faster than in mode "\textit{JFA}" due to the Multi-Thread implementation.

\begin{description}
%
\item[Null Order Statistics per Speaker] this matrix contains cumulated zero order Baum-Welch statistics for each file. Dimensions of this matrix are number of files $\times$ number of distributions of the UBM.
%
\item[First Order Statistics per Speaker] this matrix contains cumulated zero order Baum-Welch statistics for each file. Dimensions of this matrix are number of files $\times$ dimension of super vector (which is equal to the number of distributions of the UBM multiplied by the dimension of acoustic features).
\end{description}

\newpage
\subsection{TrainTarget options for i-Vectors extraction}

\begin{table}[!h]\footnotesize
\begin{center}
\begin{tabular}{m{4cm}m{4cm}m{7cm}}
Option & Value & function \tabularnewline
\hline
\hline
numThread & integer & number of thread to run in parallel. Only available if LIA\_RAL is compiled with pThread library (see ALIZE FAQ for more information) \tabularnewline
featureServerMemAlloc & integer & Specify the memory to allocate for the ALIZE buffer \tabularnewline
featureServerMode & FEATURE$\_$WRITABLE & Specify the read/write mode of the featureServer (read-only by default) \tabularnewline
featureServerBufferSize & ALL$\_$FEATURES & Algorithm Buffer \tabularnewline
minLLK & integer & Specify minimum likelihood values \tabularnewline
maxLLK & integer & Specify maximum likelihood values \tabularnewline
MAPAlgo & string & not used in \textit{iVector} mode but need to be fill\tabularnewline
\hline
matrixFilesPath & path & directory to load and save matrices model, could be relative or absolute \tabularnewline
mixtureFilesPath & path & directory to load GMM model, could be relative or absolute \tabularnewline
vectorFilesPath & path & directory to store i-Vector files, could be relative or absolute\tabularnewline
\hline
loadMixtureFileFormat & RAW | XML & format of the GMM to load\tabularnewline
loadMatrixFormat & DB | DT & format of the matrix to load, could be binary (DB) or text (DT)\tabularnewline
saveMatrixFormat & DB | DT & format of the matrix to save, could be binary (DB) or text (DT)\tabularnewline
\hline
loadMixtureFileExtension & string & extension of GMM files to load\tabularnewline
loadMatrixFilesExtension & string & extension of Matrices files to load \tabularnewline
saveMatrixFilesExtension & string & extension of Matrices files to save\tabularnewline
vectorFilesExtension & string & extension of i-vector files \tabularnewline
\hline
channelCompensation & iVector & must be iVector to set this mode\tabularnewline
ndxFilename & string & name of the index file (NDX) containing the list of files to process\tabularnewline
inputWorldFilename & string & name of the Universal Background Model to be used for statistics computation.\tabularnewline
minDivergence & boolean & use minimum divergence reestimated mean to compute iVectors\tabularnewline
meanEstimate & string & name of the estimate of mean after minimum divergence \tabularnewline
nullOrderStatSpeaker & string & name of the file containing null order Baum-Welch statistics per file\tabularnewline
firstOrderStatSpeaker & string & name of the file containing first order Baum-Welch statistics per file\tabularnewline
totalVariabilityMatrix & string & name of the TotalVariability matrix file to load \tabularnewline
\hline
totalVariabilityNumber & integer & rank of the TotalVariability matrix \tabularnewline
labelSelectedFrames & string & feature label to use for statistics computation\tabularnewline
frameLength & float & period of the frame extraction given in seconds\tabularnewline
\hline
\end{tabular}
\end{center}
\end{table}

\section{First run}
%
Example of TotalVariability matrix estimation and i-Vector extraction is provided in \textit{IV.sh}.


\bibliographystyle{apalike-fr}
\bibliography{biblio}

\end{document}